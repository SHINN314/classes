\section{プロセス管理}
\subsection*{fork}
プログラムの実行中に子プロセスとして自分の分身のプロセスを生み出すシステムコール。
forkで呼び出されたプロセスは親プロセスと同じようにプログラムの続きから処理を行う。

\subsection*{wait}
forkで子プロセスを呼び出した時、適切にwait関数を呼び出すことで、特定の子プロセスの実行が終わるまで親プロセスの実行を停止する関数。

\subsection*{exec}
プロセスの実行するものを変更する関数。
exec関数を呼び出すことで、プロセスの実行を自分が指定したプログラムに化かすことができる。

\subsection*{setjmpとlongjmp}
\subsubsection*{setjmp}
setjmpでは引数に取った変数にsetjmpを呼び出したときのCPUのレジスタの情報やスタックポインタを格納し、セーブポイントを作成する関数。
返り値は、基本0で、lognjmpによってジャンプしてから呼び出されたときのみ返り値は変化する(後述)。

\subsubsection*{longjmp}
longjmpではenvとvalを引数とし、関数が呼び出された時にenvに保存されたレジスタ/スタックポインタの情報をもとにプログラムの実行状態にジャンプする。
また、valはプログラムのセーブ地点にジャンプした時に、setjmpの返り値をvalにするための引数である。
ただし、valを0に指定しても、longjmpを使ってジャンプしてsetjmpを呼び出したときのsetjmpの返り値は1となる。
