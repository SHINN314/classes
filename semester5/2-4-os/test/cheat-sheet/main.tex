\documentclass[uplatex,dvipdfmx,a4paper,8pt]{jsarticle}
\usepackage{graphicx}
\usepackage{amsmath}
\usepackage{latexsym}
\usepackage{multirow}
\usepackage{url}
\usepackage[separate-uncertainty]{siunitx}
\usepackage{physics}
\usepackage{enumerate}
\usepackage{bm}
\usepackage{pdfpages}
\usepackage{pxchfon}
\usepackage{tikz}
\usepackage{float}
\usepackage{listings}
\usepackage{multicol}

% lstlistingのsetting
\lstset{
    basicstyle={\tiny\ttfamily},
    identifierstyle={\tiny},
    commentstyle={\tiny\itshape},
    keywordstyle={\tiny\bfseries},
    ndkeywordstyle={\tiny},
    stringstyle={\tiny\ttfamily},
    frame={tb},
    breaklines=true,
    columns=[l]{fullflexible},
    numbers=left,
    xrightmargin=0zw,
    xleftmargin=1zw,
    numberstyle={\scriptsize},
    stepnumber=1,
    numbersep=0.5zw,
    lineskip=-0.5ex
}

% tikz setting
\usepackage{tikz}
\usetikzlibrary{automata, intersections, calc, arrows, positioning, arrows.meta}

% theories setting (for japanese language)
\usepackage{amsmath}
\usepackage{amsthm}

\theoremstyle{definition}
\newtheorem{thm}{定理}[section]
\newtheorem{lem}[thm]{補題}
\newtheorem{prop}[thm]{命題}
\newtheorem{cor}[thm]{系}
\newtheorem{ass}[thm]{仮定}
\newtheorem{conj}[thm]{予想}
\newtheorem{dfn}[thm]{定義}
\newtheorem{rem}[thm]{注}

\newtheorem*{thm*}{定理}
\newtheorem*{lem*}{補題}
\newtheorem*{prop*}{命題}
\newtheorem*{cor*}{系}
\newtheorem*{ass*}{仮定}
\newtheorem*{conj*}{予想}
\newtheorem*{dfn*}{定義}
\newtheorem*{rem*}{注}

% \renewcommand{\rmdefault}{pplj}
% \renewcommand{\sfdefault}{phv}

\setlength{\textwidth}{190mm}
\setlength{\marginparwidth}{10mm}
\setlength{\textheight}{265mm}
\setlength{\topmargin}{-15mm}
\setlength{\oddsidemargin}{-10mm}
\setlength{\evensidemargin}{-10mm}
\setlength{\parindent}{0pt}
\setlength{\parskip}{1pt}
\setlength{\columnsep}{3mm}

\def\vector#1{\mbox{\boldmath $#1$}}
\newcommand{\AmSLaTeX}{%
 $\mathcal A$\lower.4ex\hbox{$\!\mathcal M\!$}$\mathcal S$-\LaTeX}
\newcommand{\PS}{{\scshape Post\-Script}}
\def\BibTeX{{\rmfamily B\kern-.05em{\scshape i\kern-.025em b}\kern-.08em
 T\kern-.1667em\lower.7ex\hbox{E}\kern-.125em X}}
\newcommand{\DeLta}{{\mit\Delta}}
\renewcommand{\d}{{\rm d}}
\def\wcaption#1{\caption[]{\parbox[t]{100mm}{#1}}}
\def\rm#1{\mathrm{#1}}
\def\tempC{^\circ \rm{C}}

\makeatletter
\def\section{\@startsection {section}{1}{\z@}{-2ex plus -0.5ex minus -.1ex}{1ex plus .1ex}{\normalsize\bf}}
\def\subsection{\@startsection {subsection}{2}{\z@}{-1.5ex plus -0.5ex minus -.1ex}{0.5ex plus .1ex}{\small\bf}}
\def\subsubsection{\@startsection {subsubsection}{3}{\z@}{-1ex plus -0.5ex minus -.1ex}{0.3ex plus .1ex}{\footnotesize\bf}}
\makeatother

\makeatletter
\def\@seccntformat#1{\@ifundefined{#1@cntformat}%
   {\csname the#1\endcsname\quad}%      default
   {\csname #1@cntformat\endcsname}%    enable individual control
}

% proof enviroment
\renewenvironment{proof}[1][\proofname]{\par
  \pushQED{\qed}%
  \normalfont \topsep6\p@\@plus6\p@\relax
  \trivlist
  \item\relax
  {\bfseries
  #1\@addpunct{.}}\hspace\labelsep\ignorespaces
}{%
  \popQED\endtrivlist\@endpefalse
}
\makeatother

\newcommand{\tenexp}[2]{#1\times10^{#2}}


\begin{document}
\small % 基本の文字サイズを小さく
\begin{multicols*}{4}
% 列のバランスを取らずに順次埋める

\section{問1}
\hspace{1em}\(\hat{\beta_0}\)が不偏推定量であることを示すには\(\mathbb{E}[\hat{\beta_0}] = \beta_0\)であることを示せば良い。
以下、この式が成立することを証明する。

\begin{align}
    \mathbb{E}[\hat{\beta_0}]   &= \frac{1}{n}\sum_{i=1}^{n}\mathbb{E}[y_i] - \mathbb{E}[\hat{\beta_1} \overline{x}] \ \text{(\(\hat{\beta_0} = y_i - (\hat{\beta_1} x_i + \varepsilon_i)\)であることを利用)} \\
                                &= \frac{1}{n}\sum_{i=1}^{n}(\beta_0 + \beta_1 x_i) - \frac{1}{n}\sum_{i=1}^{n}(\mathbb{E}[\beta_1 x_i] + \mathbb{E}[(\sum_{j=1}^{n}w_j \varepsilon_j)x_i]) \ \text{(\(\hat{\beta_1} = \beta_1 + \sum_{i=1}^{n}w_i \varepsilon_i\)であることを利用)} \\
                                &= \beta_0 + \beta_1 \overline{x} - \beta_1 \overline{x} \label{equation_mean_rho_3} \\
                                &= \beta_0
\end{align}

\noindent ここで、式\ref{equation_mean_rho_3}では\(w_j\)と\(x_i\)が定数であることを利用して\(\sum_{j=1}^{n}w_j x_i \mathbb{E}[\varepsilon_j] = 0\)と式変形した。
以上より\(\hat{\beta_0}\)の不偏性が証明された。
\documentclass[uplatex,dvipdfmx,a4paper,10pt]{jsarticle}
\usepackage{graphicx}
\usepackage{amsmath}
\usepackage{latexsym}
\usepackage{multirow}
\usepackage{url}
\usepackage[separate-uncertainty]{siunitx}
\usepackage{physics}
\usepackage{enumerate}
\usepackage{bm}
\usepackage{pdfpages}
\usepackage{pxchfon}
\usepackage{tikz}
\usepackage{float}
\usepackage{listings}
\usepackage{amsfonts}

% lstlistingのsetting
\lstset{
    basicstyle={\ttfamily},
    identifierstyle={\small},
    commentstyle={\smallitshape},
    keywordstyle={\small\bfseries},
    ndkeywordstyle={\small},
    stringstyle={\small\ttfamily},
    frame={tb},
    breaklines=true,
    columns=[l]{fullflexible},
    numbers=left,
    xrightmargin=0zw,
    xleftmargin=3zw,
    numberstyle={\scriptsize},
    stepnumber=1,
    numbersep=1zw,
    lineskip=-0.5ex
}

% tikz setting
\usepackage{tikz}
\usetikzlibrary{automata, intersections, calc, arrows, positioning, arrows.meta}

% theories setting (for japanese language)
\usepackage{amsmath}
\usepackage{amsthm}

\theoremstyle{definition}
\newtheorem{thm}{定理}[section]
\newtheorem{lem}[thm]{補題}
\newtheorem{prop}[thm]{命題}
\newtheorem{cor}[thm]{系}
\newtheorem{ass}[thm]{仮定}
\newtheorem{conj}[thm]{予想}
\newtheorem{dfn}[thm]{定義}
\newtheorem{rem}[thm]{注}

\newtheorem*{thm*}{定理}
\newtheorem*{lem*}{補題}
\newtheorem*{prop*}{命題}
\newtheorem*{cor*}{系}
\newtheorem*{ass*}{仮定}
\newtheorem*{conj*}{予想}
\newtheorem*{dfn*}{定義}
\newtheorem*{rem*}{注}

% \renewcommand{\rmdefault}{pplj}
% \renewcommand{\sfdefault}{phv}

\setlength{\textwidth}{165mm} %165mm-marginparwidth
\setlength{\marginparwidth}{40mm}
\setlength{\textheight}{225mm}
\setlength{\topmargin}{-5mm}
\setlength{\oddsidemargin}{-3.5mm}
% \setlength{\parindent}{0pt}

\def\vector#1{\mbox{\boldmath $#1$}}
\newcommand{\AmSLaTeX}{%
 $\mathcal A$\lower.4ex\hbox{$\!\mathcal M\!$}$\mathcal S$-\LaTeX}
\newcommand{\PS}{{\scshape Post\-Script}}
\def\BibTeX{{\rmfamily B\kern-.05em{\scshape i\kern-.025em b}\kern-.08em
 T\kern-.1667em\lower.7ex\hbox{E}\kern-.125em X}}
\newcommand{\DeLta}{{\mit\Delta}}
\renewcommand{\d}{{\rm d}}
\def\wcaption#1{\caption[]{\parbox[t]{100mm}{#1}}}
\def\rm#1{\mathrm{#1}}
\def\tempC{^\circ \rm{C}}

\makeatletter
\def\section{\@startsection {section}{1}{\z@}{-3.5ex plus -1ex minus -.2ex}{2.3ex plus .2ex}{\Large\bf}}
\def\subsection{\@startsection {subsection}{2}{\z@}{-3.25ex plus -1ex minus -.2ex}{1.5ex plus .2ex}{\normalsize\bf}}
\def\subsubsection{\@startsection {subsubsection}{3}{\z@}{-3.25ex plus -1ex minus -.2ex}{1.5ex plus .2ex}{\small\bf}}
\makeatother

\makeatletter
\def\@seccntformat#1{\@ifundefined{#1@cntformat}%
   {\csname the#1\endcsname\quad}%      default
   {\csname #1@cntformat\endcsname}%    enable individual control
}

% proof enviroment
\renewenvironment{proof}[1][\proofname]{\par
  \pushQED{\qed}%
  \normalfont \topsep6\p@\@plus6\p@\relax
  \trivlist
  \item\relax
  {\bfseries
  #1\@addpunct{.}}\hspace\labelsep\ignorespaces
}{%
  \popQED\endtrivlist\@endpefalse
}
\makeatother

\newcommand{\tenexp}[2]{#1\times10^{#2}}


\begin{document}
% タイトル
\begin{center}
{\Large{\bf 経済学B 第2講課題}} \\
{\bf 電気通信大学 Ⅰ類 コンピュータサイエンスプログラム 3年} \\
{\bf 2311081 木村慎之介} \\
\end{center}

\section{設問1}
\hspace{1em}産出量ベクトル\(X\)、最終需要ベクトル\(F\)、投入係数行列\(A\)の間には以下の関係式が成立している。

\begin{align}
    X   &= AX + F \\
    \Delta X    &= A\Delta X + \Delta F
\end{align}

\noindent 以上の式を用いて雇用量の変化量を求めた。
まず、投入係数行列を計算した。
計算結果は以下のようになった。

\begin{equation}
    A   = \begin{bmatrix}
        0.5     & 0.15 \\
        0.4     & 0.45
    \end{bmatrix}
\end{equation}

\noindent 次に問題分から最終需要量の変化ベクトル\(\Delta F\)を求めると以下のようになった。

\begin{equation}
    \Delta F = \begin{bmatrix}
        10 \\
        0
    \end{bmatrix}
\end{equation}

\noindent 以上のベクトル、行列と関係式を用いて算出力の変化ベクトルを求めると以下のようになった。

\begin{align}
    \Delta X    &= (I - A)^{-1}\Delta F \\
                &= \frac{1}{43}\begin{bmatrix}
                    110     & 30 \\
                    80      & 100 \\
                \end{bmatrix}
                \begin{bmatrix}
                    10 \\
                    0
                \end{bmatrix} \\
                &= \frac{1}{43} \begin{bmatrix}
                    1100 \\
                    800
                \end{bmatrix}
\end{align}

\noindent 次に雇用係数\(L\)を求めた。
求めた結果は以下のようになった。

\begin{equation}
    L = (l_1 \ l_2) = (0.4 \ 0.525)
\end{equation}

\noindent 以上の結果を用いて経済全体の雇用料の変化\(\Delta E\)を求めると以下のようになった。

\begin{align}
    \Delta E    &= L\Delta X \\
                &= 20
\end{align}

\section{設問2}
\subsection{ラスパイレス指数について}
\hspace{1em}ラスパイレス指数とは、過去のある年の購入数量を固定したウェイトとして使用する指数である。
この指数では基準年と同じ量のものを、現在の価格で買ったときの値段を示す。
特徴としては、消費者の代替効果を考慮しないため、実際の物価変動より高めに評価される傾向にある。
消費者物価指数などに使われている。

\subsection{パーシェ指数}
\hspace{1em}パーシェ指数とは現在の購入数量をウェイトとして使用する指数である。
この指数では比較年と同じものを過去の価格で買うといくらになったかを示す。
特徴としてはウェイトが毎年変わるため計算に手間がかかってしまう。
また、代替効果の影響を受けやすく物価変動を実際より低めに評価される傾向にある。
GDPデフレータなどに使われる。

%%%%%%%%%%%%%%%%%%%%%%%%%%%%%%%%%%%%%%%%%%%%%%%%%%%%%%%%%%%%%%%%%%%%%%
\appendix
\setcounter{figure}{0}
\setcounter{table}{0}
\numberwithin{equation}{section}
\renewcommand{\thetable}{\Alph{section}\arabic{table}}
\renewcommand{\thefigure}{\Alph{section}\arabic{figure}}
%\def\thesection{付録\Alph{section}}
\makeatletter 
\newcommand{\section@cntformat}{付録 \thesection:\ }
\makeatother
%%%%%%%%%%%%%%%%%%%%%%%%%%%%%%%%%%%%%%%%%%%%%%%%%%%%%%%%%%%%%%%%%%%%%%

    
\end{document}
\documentclass[uplatex,dvipdfmx,a4paper,10pt]{jsarticle}
\usepackage{graphicx}
\usepackage{amsmath}
\usepackage{latexsym}
\usepackage{multirow}
\usepackage{url}
\usepackage[separate-uncertainty]{siunitx}
\usepackage{physics}
\usepackage{enumerate}
\usepackage{bm}
\usepackage{pdfpages}
\usepackage{pxchfon}
\usepackage{tikz}
\usepackage{float}
\usepackage{listings}
\usepackage{amsfonts}
\usepackage{empheq}

% lstlistingのsetting
\lstset{
    basicstyle={\ttfamily},
    identifierstyle={\small},
    commentstyle={\smallitshape},
    keywordstyle={\small\bfseries},
    ndkeywordstyle={\small},
    stringstyle={\small\ttfamily},
    frame={tb},
    breaklines=true,
    columns=[l]{fullflexible},
    numbers=left,
    xrightmargin=0zw,
    xleftmargin=3zw,
    numberstyle={\scriptsize},
    stepnumber=1,
    numbersep=1zw,
    lineskip=-0.5ex
}

% tikz setting
\usepackage{tikz}
\usetikzlibrary{automata, intersections, calc, arrows, positioning, arrows.meta}

% theories setting (for japanese language)
\usepackage{amsmath}
\usepackage{amsthm}

\theoremstyle{definition}
\newtheorem{thm}{定理}[section]
\newtheorem{lem}[thm]{補題}
\newtheorem{prop}[thm]{命題}
\newtheorem{cor}[thm]{系}
\newtheorem{ass}[thm]{仮定}
\newtheorem{conj}[thm]{予想}
\newtheorem{dfn}[thm]{定義}
\newtheorem{rem}[thm]{注}

\newtheorem*{thm*}{定理}
\newtheorem*{lem*}{補題}
\newtheorem*{prop*}{命題}
\newtheorem*{cor*}{系}
\newtheorem*{ass*}{仮定}
\newtheorem*{conj*}{予想}
\newtheorem*{dfn*}{定義}
\newtheorem*{rem*}{注}

% \renewcommand{\rmdefault}{pplj}
% \renewcommand{\sfdefault}{phv}

\setlength{\textwidth}{165mm} %165mm-marginparwidth
\setlength{\marginparwidth}{40mm}
\setlength{\textheight}{225mm}
\setlength{\topmargin}{-5mm}
\setlength{\oddsidemargin}{-3.5mm}
% \setlength{\parindent}{0pt}

\def\vector#1{\mbox{\boldmath $#1$}}
\newcommand{\AmSLaTeX}{%
 $\mathcal A$\lower.4ex\hbox{$\!\mathcal M\!$}$\mathcal S$-\LaTeX}
\newcommand{\PS}{{\scshape Post\-Script}}
\def\BibTeX{{\rmfamily B\kern-.05em{\scshape i\kern-.025em b}\kern-.08em
 T\kern-.1667em\lower.7ex\hbox{E}\kern-.125em X}}
\newcommand{\DeLta}{{\mit\Delta}}
\renewcommand{\d}{{\rm d}}
\def\wcaption#1{\caption[]{\parbox[t]{100mm}{#1}}}
\def\rm#1{\mathrm{#1}}
\def\tempC{^\circ \rm{C}}

\makeatletter
\def\section{\@startsection {section}{1}{\z@}{-3.5ex plus -1ex minus -.2ex}{2.3ex plus .2ex}{\Large\bf}}
\def\subsection{\@startsection {subsection}{2}{\z@}{-3.25ex plus -1ex minus -.2ex}{1.5ex plus .2ex}{\normalsize\bf}}
\def\subsubsection{\@startsection {subsubsection}{3}{\z@}{-3.25ex plus -1ex minus -.2ex}{1.5ex plus .2ex}{\small\bf}}
\makeatother

\makeatletter
\def\@seccntformat#1{\@ifundefined{#1@cntformat}%
   {\csname the#1\endcsname\quad}%      default
   {\csname #1@cntformat\endcsname}%    enable individual control
}

% proof enviroment
\renewenvironment{proof}[1][\proofname]{\par
  \pushQED{\qed}%
  \normalfont \topsep6\p@\@plus6\p@\relax
  \trivlist
  \item\relax
  {\bfseries
  #1\@addpunct{.}}\hspace\labelsep\ignorespaces
}{%
  \popQED\endtrivlist\@endpefalse
}
\makeatother

\newcommand{\tenexp}[2]{#1\times10^{#2}}


\begin{document}
% タイトル
\begin{center}
{\Large{\bf 経済学B 第3講課題}} \\
{\bf 電気通信大学 Ⅰ類 コンピュータサイエンスプログラム 3年} \\
{\bf 2311081 木村慎之介} \\
\end{center}

\section{問題1}
\hspace{1em}条件\(G = 50, B = 20\)と以下の式をもとに税率\(t\)を求めた。

\begin{empheq}[left=\empheqlbrace]{align}
    Y   &= C + I + G \\
    C   &= 0.9(Y - T) \\
    I   &= 41 \\
    T   &= tY \\
    T   &= G + B \\
\end{empheq}

\noindent まず条件と式(5)から

\begin{align}
    T   &= G + B \\
    T   &= 70
\end{align}

\noindent と求まる。
次に求めた\(T\)の値と式(1)(2)を用いて\(Y\)を求めた。
そのために\(Y\)を\(C\)の関数として表した。

\begin{equation}
    C = 0.9(Y - T) = 0.9(Y - 70)
\end{equation}

\noindent 以上の式を(1)に代入すると次のようになる。

\begin{align}
    Y   &= 0.9(Y - 70) + 41 + 50 \\
    Y   &= 280
\end{align}

\noindent 最後に式(4)に代入して税率を求めると以下のようになった。

\begin{align}
    70  &= t \times 280 \\
    t   &= 0.25
\end{align}

\section{問題2}
\hspace{1em}まず与えられた条件をした。
整理した結果は以下のようになった。

\begin{empheq}[left=\empheqlbrace]{align}
    Y   &= 400 \\
    b   &= 0.8 \\
    a   &= 20 \\
    I   &= 40 \\
    T   &= tY \\
    T   &= G \\
    Y   &= C + I + G
\end{empheq}

\noindent まず式(14)(18)(19)から税収と政府支出を以下のように求めた。

\begin{equation}
    T = G = 400t
\end{equation}

\noindent 次に消費関数が\(C = a + b(Y - T)\)であることから\(C\)に\(a, b, T, Y\)を代入することで\(C\)を\(t\)の式で以下のように表した。

\begin{align}
    C   &= 20 + 0.8(400 - 400t) \\
    C   &= 340 -320t
\end{align}

最後に式(20)と今までに求めた値を用いることで以下のように税率を求めた。

\begin{align}
    400     &= (340 - 320t) + 40 + 400t \\
    t       &= 0.25
\end{align}

%%%%%%%%%%%%%%%%%%%%%%%%%%%%%%%%%%%%%%%%%%%%%%%%%%%%%%%%%%%%%%%%%%%%%%
\appendix
\setcounter{figure}{0}
\setcounter{table}{0}
\numberwithin{equation}{section}
\renewcommand{\thetable}{\Alph{section}\arabic{table}}
\renewcommand{\thefigure}{\Alph{section}\arabic{figure}}
%\def\thesection{付録\Alph{section}}
\makeatletter 
\newcommand{\section@cntformat}{付録 \thesection:\ }
\makeatother
%%%%%%%%%%%%%%%%%%%%%%%%%%%%%%%%%%%%%%%%%%%%%%%%%%%%%%%%%%%%%%%%%%%%%%

    
\end{document}
\section{処理装置(CPU)の管理}
\subsection*{スケジューリング評価基準}
\begin{itemize}
    \item スループット: 単位時間あたりの仕事量
    \item CPU使用率: CPU動作時間 / システム稼働時間
    \item 待ち時間: ジョブ完了までに実行可キューで待つ時間の合計
    \item 応答時間: ジョブの要求から最初の応答までの時間
\end{itemize}

\subsection*{スケジューリングアルゴリズム}
\subsubsection*{到着順}
単純なFIFO。
\begin{itemize}
    \item メリット: オーバヘッド(付加的な処理)が少ない
    \item デメリット: 性能は良くない
\end{itemize}

\subsubsection*{最短時間順(処理時間順)}
ジョブをキューに追加する時、処理時間が小さいものを優先するようにする優先度付きFIFO。
\begin{itemize}
    \item メリット: 平均待ち時間が短い
    \item デメリット: 処理時間を予めわかっていないと行けない
\end{itemize}

\subsubsection*{優先度順}
プロセスごとに優先度をつける(CPU処理時間の場合は最短時間順)。

\begin{itemize}
    \item デメリット: 飢餓に陥ることがある
\end{itemize}

\subsubsection*{最小残余時間優先}
新しく到着したジョブの実行時間が、実行中のジョブの残り時間より小さい場合、到着したジョブを優先して実行するアルゴリズム。
キューに追加する場合は処理順になるようにキューをソートする。

\subsubsection*{ラウンドロビン}
一定の時間感覚(タイムスライス)でジョブをCPUに割り当てる。
実行が変わったら、キューの末尾に追加される。
\section{記憶管理}
\subsection*{関数ポインタとセグフォ}
関数ポインタは読み込みのみが許されている記憶領域に保存されている。
そのためプログラム内でポインタを読み込むことはできるが、書き込もうとするとosによってセグメンテーション違反としてエラーが吐き出される。
\subsection*{仮想アドレス \(\backslash\) 論理アドレス}
アプリケーションがメモリにアクセスする時に使うアドレス。
実アドレスと一致するとは限らない。

\subsection*{実アドレス \(\backslash\) 物理アドレス}
実際の物理メモリ上のアドレス。

\subsection*{MMU}
仮想アドレスと実アドレスを対応付けるOSの機能。
ブロック単位で対応をつける。
\begin{itemize}
    \item ブロックサイズが固定: ページング
    \item ブロックサイズが可変: セグメンテーション
\end{itemize}
\section{割り込み処理と入出力処理}
\subsection*{signal}
シグナル処理を行うための関数(割り込み/非同期関数)。
第一引数にシグナルの種類、第二引数に割り込み時に実行する関数を指定する。

\subsection*{alarm}
指定した時間の後にアラームシグナル(SIGALRM)を割り込み処理として送る関数。

\subsection*{タイマ割り込み}
一定時間ごとに非同期に発生する処理。
タイマ割り込みはUNIX時間の更新に使われる。
%%%%%%%%%%%%%%%%%%%%%%%%%%%%%%%%%%%%%%%%%%%%%%%%%%%%%%%%%%%%%%%%%%%%%%
\appendix
\setcounter{figure}{0}
\setcounter{table}{0}
\numberwithin{equation}{section}
\renewcommand{\thetable}{\Alph{section}\arabic{table}}
\renewcommand{\thefigure}{\Alph{section}\arabic{figure}}
%\def\thesection{付録\Alph{section}}
\makeatletter 
\newcommand{\section@cntformat}{付録 \thesection:\ }
\makeatother
%%%%%%%%%%%%%%%%%%%%%%%%%%%%%%%%%%%%%%%%%%%%%%%%%%%%%%%%%%%%%%%%%%%%%%

\end{multicols*}
\end{document}