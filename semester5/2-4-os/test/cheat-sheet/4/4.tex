\section{処理装置(CPU)の管理}
\subsection*{スケジューリング評価基準}
\begin{itemize}
    \item スループット: 単位時間あたりの仕事量
    \item CPU使用率: CPU動作時間 / システム稼働時間
    \item 待ち時間: ジョブ完了までに実行可キューで待つ時間の合計
    \item 応答時間: ジョブの要求から最初の応答までの時間
\end{itemize}

\subsection*{スケジューリングアルゴリズム}
\subsubsection*{到着順}
単純なFIFO。
\begin{itemize}
    \item メリット: オーバヘッド(付加的な処理)が少ない
    \item デメリット: 性能は良くない
\end{itemize}

\subsubsection*{最短時間順(処理時間順)}
ジョブをキューに追加する時、処理時間が小さいものを優先するようにする優先度付きFIFO。
\begin{itemize}
    \item メリット: 平均待ち時間が短い
    \item デメリット: 処理時間を予めわかっていないと行けない
\end{itemize}

\subsubsection*{優先度順}
プロセスごとに優先度をつける(CPU処理時間の場合は最短時間順)。

\begin{itemize}
    \item デメリット: 飢餓に陥ることがある
\end{itemize}

\subsubsection*{最小残余時間優先}
新しく到着したジョブの実行時間が、実行中のジョブの残り時間より小さい場合、到着したジョブを優先して実行するアルゴリズム。
キューに追加する場合は処理順になるようにキューをソートする。

\subsubsection*{ラウンドロビン}
一定の時間感覚(タイムスライス)でジョブをCPUに割り当てる。
実行が変わったら、キューの末尾に追加される。