\documentclass[uplatex,dvipdfmx,a4paper,10pt]{jsarticle}
\usepackage{graphicx}
\usepackage{amsmath}
\usepackage{latexsym}
\usepackage{multirow}
\usepackage{url}
\usepackage[separate-uncertainty]{siunitx}
\usepackage{physics}
\usepackage{enumerate}
\usepackage{bm}
\usepackage{pdfpages}
\usepackage{pxchfon}
\usepackage{tikz}
\usepackage{float}
\usepackage{listings}
\usepackage{amsthm}

% lstlistingのsetting
\lstset{
    basicstyle={\ttfamily},
    identifierstyle={\small},
    commentstyle={\smallitshape},
    keywordstyle={\small\bfseries},
    ndkeywordstyle={\small},
    stringstyle={\small\ttfamily},
    frame={tb},
    breaklines=true,
    columns=[l]{fullflexible},
    numbers=left,
    xrightmargin=0zw,
    xleftmargin=3zw,
    numberstyle={\scriptsize},
    stepnumber=1,
    numbersep=1zw,
    lineskip=-0.5ex
}

% tikz setting
\usepackage{tikz}
\usetikzlibrary{automata, intersections, calc, arrows, positioning, arrows.meta}

% theories setting (for japanese language)
\usepackage{amsmath}
\usepackage{amsthm}

\theoremstyle{definition}
\newtheorem{thm}{定理}[section]
\newtheorem{lem}[thm]{補題}
\newtheorem{prop}[thm]{命題}
\newtheorem{cor}[thm]{系}
\newtheorem{ass}[thm]{仮定}
\newtheorem{conj}[thm]{予想}
\newtheorem{dfn}[thm]{定義}
\newtheorem{rem}[thm]{注}

\newtheorem*{thm*}{定理}
\newtheorem*{lem*}{補題}
\newtheorem*{prop*}{命題}
\newtheorem*{cor*}{系}
\newtheorem*{ass*}{仮定}
\newtheorem*{conj*}{予想}
\newtheorem*{dfn*}{定義}
\newtheorem*{rem*}{注}

% proof enviroment

\setlength{\textwidth}{165mm} %165mm-marginparwidth
\setlength{\marginparwidth}{40mm}
\setlength{\textheight}{225mm}
\setlength{\topmargin}{-5mm}
\setlength{\oddsidemargin}{-3.5mm}

\def\vector#1{\mbox{\boldmath $#1$}}
\newcommand{\AmSLaTeX}{%
 $\mathcal A$\lower.4ex\hbox{$\!\mathcal M\!$}$\mathcal S$-\LaTeX}
\newcommand{\PS}{{\scshape Post\-Script}}
\def\BibTeX{{\rmfamily B\kern-.05em{\scshape i\kern-.025em b}\kern-.08em
 T\kern-.1667em\lower.7ex\hbox{E}\kern-.125em X}}
\newcommand{\DeLta}{{\mit\Delta}}
\renewcommand{\d}{{\rm d}}
\def\wcaption#1{\caption[]{\parbox[t]{100mm}{#1}}}
\def\rm#1{\mathrm{#1}}
\def\tempC{^\circ \rm{C}}

\makeatletter
\def\section{\@startsection {section}{1}{\z@}{-3.5ex plus -1ex minus -.2ex}{2.3ex plus .2ex}{\Large\bf}}
\def\subsection{\@startsection {subsection}{2}{\z@}{-3.25ex plus -1ex minus -.2ex}{1.5ex plus .2ex}{\normalsize\bf}}
\def\subsubsection{\@startsection {subsubsection}{3}{\z@}{-3.25ex plus -1ex minus -.2ex}{1.5ex plus .2ex}{\small\bf}}
\makeatother

\makeatletter
\def\@seccntformat#1{\@ifundefined{#1@cntformat}%
   {\csname the#1\endcsname\quad}%      default
   {\csname #1@cntformat\endcsname}%    enable individual control
}

% reconstruct proof enviroment
\renewenvironment{proof}[1][\proofname]{\par
  \pushQED{\qed}%
  \normalfont \topsep6\p@\@plus6\p@\relax
  \trivlist
  \item\relax
  {\bfseries
  #1\@addpunct{.}}\hspace\labelsep\ignorespaces
}{%
  \popQED\endtrivlist\@endpefalse
}
\makeatother

\newcommand{\tenexp}[2]{#1\times10^{#2}}


\begin{document}
% タイトル
\begin{center}
{\Large{\bf グラフとネットワーク第5回課題}} \\
{\bf 電気通信大学 Ⅰ類 コンピュータサイエンスプログラム 3年} \\
{\bf 2311081 木村慎之介} \\
\end{center}

\section{問1}
\hspace{1em}問1では以下の命題を証明する。

\begin{prop}
グラフ\(T = (V, E)\)について、\(T\)は閉路を持たずかつ\(|E| = |V| - 1\)であるならば、\(T\)は連結でありかつ\(|E| = |V| - 1\)である。
\label{prop_two_to_three}
\end{prop}

\begin{proof}[\textbf{命題\ref{prop_two_to_three}の証明}]
グラフ\(T = (E, V)\)が閉路を持たず、\(|E| = |V| - 1\)を満たすことを仮定する。
以降は背理法を用いて証明する。ただし、\(|E| \neq |V| - 1\)を仮定すると矛盾が生じることは最初の仮定から自明にわかるので省略する。 \\
\hspace{1em}\(T\)が連結でないと仮定する。すると、最初の仮定から\(T\)は閉路を持たないので\(T\)が森であることがわかる。
さて、グラフ\(T\)が森であることからグラフ\(T\)には\(n\)個の木が含まれる(ただし\(n \geq 2\))。
ここで木\(T' = (E', V')\)について\(|E'| = |V'| - 1\)が成立することから、\(T\)に含まれる\(i\)番目の木を\(T_i = (E_i, V_i)\)と表すと、\(|E|\)について次の式が成立する。
\begin{align}
  |E| &= \sum_{i=1}^{n}|E_i| \\
      &= \sum_{i=1}^{n}(|V_i| - 1) (\text{木の性質から}) \\
      &= \sum_{i=1}^{n}|V_i| - n \\
      &= |V| - n
\end{align}
しかし、この式は最初に仮定した「グラフ\(T\)について\(|E| = |V| - 1\)が成立する」ということに矛盾する。従ってグラフ\(T\)が連結であることが証明された。
\end{proof}

\section{問2}
\hspace{1em}問2では以下の命題を証明する。

\begin{prop}
グラフ\(T = (E, V)\)について、\(T\)は閉路を持たずかつ隣接しない頂点に辺を加えると閉路ができるならば、グラフ\(T\)は木である(つまり、グラフ\(T\)は閉路を持たずかつ連結である)。
\label{prop_six_to_one}
\end{prop}

\begin{proof}[\textbf{命題\ref{prop_six_to_one}の証明}]
グラフ\(T = (E, V)\)について、\(T\)が閉路を持たずかつ隣接しない頂点に辺を加えると閉路ができることを仮定する。
以降は背理法を用いて証明する。
なお、グラフ\(T\)が経路を持つことを仮定すると矛盾が生じることは最初のっ仮定から自明にわかるので省略する。 \\
\hspace{1em}グラフ\(T\)が連結でないことを仮定する。
このとき、グラフ\(T\)は森となる。
さて、この森の中から2つの木\(T_1\)、\(T_2\)を持ってきて、\(T_1\)から任意に持ってきた一つの頂点\(u\)と\(T_2\)から任意に持ってきた一つの頂点\(v\)の間に辺を引く。
このように構成したグラフを\(T'\)とおく。
この\(T'\)が実は木であることを以下で証明していく。\\
\hspace{1em}\(T'\)が木であることを示すには\(T'\)の任意の2頂点を結ぶ道がただ1つしかないことを示せば良い。
以下、頂点\(x\)から頂点\(y\)への道\(x, y\)についての一意性を2パターンに分けて証明する。
\begin{enumerate}
  \item \(x, y\)が\(T_1\)の頂点である時 \\
        \hspace{1em}\(x, y\)が\(T_1\)の頂点の時、\(T_1\)が木であったことから、\(T_1\)に含まれる頂点のみを通る道\(P = u_0 u_1 \cdots u_m (u_0 = x, u_m = y)\)が存在する。
        ここで\(T_1\)が木であることから、頂点\(u, v\)をつなぐ\(P\)とは異なる道を作るには\(T_2\)に含まれる頂点を通過する必要がある。
        しかし、頂点\(x\)から木\(T_2\)の頂点を通過し頂点\(y\)へ行くには辺\((u, v)\)を二回は通過する必要がある。
        よって、\(T_2\)の頂点を用いても\(P\)とは別の道を作ることはできないため一意性が示された。
  \item \hspace{1em}\(x\)が\(T_1\)の頂点で、\(y\)が\(T_2\)の頂点である時
        \hspace{1em}\(T_1\)の頂点から\(T_2\)の頂点へ行くには辺\((u, v)\)を通過する必要がある。
        このとき\(x\)から\(y\)への道として\(P = u_0 u_1 \cdots u_m u v v_1 \cdots v_{n-1} v_n\)が取れる。
        ただし\(u_0, u_1, \cdots, u_m, u\)は木\(T_1\)の頂点で、\(v, v_1, \cdots, v_{n-1}, v_n\)は木\(T_2\)の頂点である。
        ここで、パターン1より道\(u_0 u_1 \cdots u_m u\)と道\(v v_1 \cdots v_{n-1} v_n\)は一意に決まる道であるから道\(P\)も一意に決まる。
\end{enumerate}
\hspace{1em}\(x, y\)が\(T_2\)の頂点である時と、\(x\)が木\(T_2\)の頂点で\(y\)が木\(T_1\)
よって任意の2頂点を結ぶ道は一意に決まることが示されたのでグラフ\(T'\)は木である。
しかし、これはグラフ\(T\)の隣接しない頂点に辺を追加したときに閉路ができることに矛盾する。
従ってグラフ\(T\)は連結であることが示され、\(T\)が木であることがわかった。
\end{proof}

%%%%%%%%%%%%%%%%%%%%%%%%%%%%%%%%%%%%%%%%%%%%%%%%%%%%%%%%%%%%%%%%%%%%%%
\appendix
\setcounter{figure}{0}
\setcounter{table}{0}
\numberwithin{equation}{section}
\renewcommand{\thetable}{\Alph{section}\arabic{table}}
\renewcommand{\thefigure}{\Alph{section}\arabic{figure}}
%\def\thesection{付録\Alph{section}}
\makeatletter 
\newcommand{\section@cntformat}{付録 \thesection:\ }
\makeatother
%%%%%%%%%%%%%%%%%%%%%%%%%%%%%%%%%%%%%%%%%%%%%%%%%%%%%%%%%%%%%%%%%%%%%%

    
\end{document}