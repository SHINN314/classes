\documentclass[uplatex,dvipdfmx,a4paper,10pt]{jsarticle}
\usepackage{graphicx}
\usepackage{amsmath}
\usepackage{latexsym}
\usepackage{multirow}
\usepackage{url}
\usepackage[separate-uncertainty]{siunitx}
\usepackage{physics}
\usepackage{enumerate}
\usepackage{bm}
\usepackage{pdfpages}
\usepackage{pxchfon}
\usepackage{tikz}
\usepackage{float}

% tikz setting
\usepackage{tikz}
\usetikzlibrary{automata, intersections, calc, arrows, positioning, arrows.meta}

% \renewcommand{\rmdefault}{pplj}
% \renewcommand{\sfdefault}{phv}

\setlength{\textwidth}{165mm} %165mm-marginparwidth
\setlength{\marginparwidth}{40mm}
\setlength{\textheight}{225mm}
\setlength{\topmargin}{-5mm}
\setlength{\oddsidemargin}{-3.5mm}
% \setlength{\parindent}{0pt}

\def\vector#1{\mbox{\boldmath $#1$}}
\newcommand{\AmSLaTeX}{%
 $\mathcal A$\lower.4ex\hbox{$\!\mathcal M\!$}$\mathcal S$-\LaTeX}
\newcommand{\PS}{{\scshape Post\-Script}}
\def\BibTeX{{\rmfamily B\kern-.05em{\scshape i\kern-.025em b}\kern-.08em
 T\kern-.1667em\lower.7ex\hbox{E}\kern-.125em X}}
\newcommand{\DeLta}{{\mit\Delta}}
\renewcommand{\d}{{\rm d}}
\def\wcaption#1{\caption[]{\parbox[t]{100mm}{#1}}}
\def\rm#1{\mathrm{#1}}
\def\tempC{^\circ \rm{C}}

\makeatletter
\def\section{\@startsection {section}{1}{\z@}{-3.5ex plus -1ex minus -.2ex}{2.3ex plus .2ex}{\Large\bf}}
\def\subsection{\@startsection {subsection}{2}{\z@}{-3.25ex plus -1ex minus -.2ex}{1.5ex plus .2ex}{\normalsize\bf}}
\def\subsubsection{\@startsection {subsubsection}{3}{\z@}{-3.25ex plus -1ex minus -.2ex}{1.5ex plus .2ex}{\small\bf}}
\makeatother

\makeatletter
\def\@seccntformat#1{\@ifundefined{#1@cntformat}%
   {\csname the#1\endcsname\quad}%      default
   {\csname #1@cntformat\endcsname}%    enable individual control
}
\makeatother

\newcommand{\tenexp}[2]{#1\times10^{#2}}


\begin{document}
    % タイトル
    \begin{center}
        {\Large{\bf 科学史B 学期末レポート}} \\
        {\bf 2311081 木村慎之介} \\
    \end{center}

    \section*{第2回}
      \hspace{1em}殷王朝の時代、中国では甲骨文字が使われていた。
      その甲骨文字には現在の漢数字のもととなる数字の表現も登場していた。
      戦国時代には算籌と呼ばれる計算道具を用いて四則演算や方程式、冪根の計算を行っていた。 \\
      \hspace{1em}殷王朝の時代には土器から青銅器へ移り変わった時期でもある。
      その後戦国時代になると青銅は貨幣として扱われ、中国に貨幣が流通するようになる。
      そして秦王朝の時代に文字や度量衡の統一が行われた。 \\
      \hspace{1em}以上の背景により、前漢の時代の官僚は民を統治するために数学の知識を学ぶようになった。
      官僚は九章算術と呼ばれる数学の問題集を用いて行政に必要な数学を学んでいたと推定されている。
      この九章算術は古代・中世中国の数学の規範となった。

    \section*{第3回}
    \hspace{1em}古代中国では天文観測がすでに行われており、流星群やハレー彗星を観測した記録が確認されている。
    この天文の観測は時間と空間の決定に大きく貢献した。 \\
    \hspace{1em}古代中国では十干十二支を日付に対応させ、二十八宿で位置を調べていた。
    特に二十八宿は位置を調べるのに星を基準にしていた。 \\
    \hspace{1em}さらに、ひと月の周期を定めるために太陰太陽暦を用いていた。
    太陰太陽暦は太陰暦と太陽歴の周期のずれを閏月を加えることで周期のズレを調整した暦である。
    暦は天人相関説に基づいて王朝が変わるたびに方針を変更して修正されていった。
    しかし、時代が進むに連れ天人相関説の思想は薄れ、天文観測データをもとに行われるようになった。
    その後南朝の時代になると、祖沖之が歳差を暦に導入することで精密な周期を提唱し、息子の祖暅之が大明暦として採用へ導いた。 \\
    \hspace{1em}また、古代中国では宇宙論についての記述も残っているが、当時の説には説としての妥当性は見られなかった。

    \section*{第4回}
    \hspace{1em}唐から宋に時代が変化すると中国は商業・技術革新が起きた。
    また元の時代にはモンゴルがユーラシア大陸を支配し、交通網が整備された。
    この背景のもと、中国の伝統的な数学・天文学は絶頂を迎えた。 \\
    \hspace{1em}唐の時代では算経十書とよばれる十冊の数学書により算術の勉強が行われていた。
    その中には連立方程式や剰余の問題が含まれていた。
    南宋時代になると、九章算術にならった数書九章や、方程式・魔法陣などが収録された楊輝算法が算術の教科書となった。
    元の初期には南宋末、元初期には算籌による方程式の構成法である天元術が発展した。 \\
    \hspace{1em}暦の面では、近代以前の最高精度の中国暦である授時暦が開発された。
    開発の背景には宋時代以降の観測データの信頼性の低下がある。
    時代を経て知識や技術が元に伝来したことも影響し太史院と呼ばれる暦改定部門が発足し授時暦が作成された。
    授時暦は精密な観測と独創的なデータの補完方法により高い精度を出した。(392文字)

    \section*{第5回}
    \hspace{1em}明王朝初期は歓農抑商政策を実施したが、経済が低迷してしまった。
    しかし、中期以降になると海外から銀が流入したことで銀経済となった。
    加えて商工業が展開されることになり実学が重視されるようになった。
    この時代に書かれた実学書は日本にも大きな影響を与えた。 \\
    \hspace{1em}また、明王朝の時代は西欧科学と接触した時代でもある。
    当時、大統暦が月食・日食の予報を失敗しており暦の不正確さが問題となっていた。
    そこにイエズス会が目をつけ、キリスト教布教の一環として西洋の天文や地理の知識を中国の官僚に紹介した。
    この背景から西洋天文学を導入し、改暦するといった意見が中国国内に出てきた。
    そこで、徐光啓が湯若望らを登用時憲暦の基礎となった崇禎暦書の編纂が行われた。
    崇禎暦書では望遠鏡観測の結果とティコ・ブラーエの宇宙論が導入されている。
    清王朝の時代にはイエズス会から派遣された南懐仁が楊光先に変わり多くの業務を行うようになった。
    
    \section*{第6回}
    \hspace{1em}朝鮮半島では中国の影響を受けながらも独自の文化を発展させていた。
    新羅の時代には瞻星台が建てられており、天文の観測をしていた形跡が存在する。
    また印刷術や算学制度もみられた。 \\
    \hspace{1em}高麗の時代になると宋代の影響で青磁器が展開が見られたり、地図の作成や金属活字、さらに科挙に算学を課していた様子も見られる。 \\
    \hspace{1em}李氏朝鮮、特に世宗時代ではハングル活字の開発や授時暦を適用した暦法研究をはじめとした天文学の発展、さらに時計の開発がなされた。
    世宗以降の時代では兩班と中人による知識独占が行われた。
    しかし、中人は天元術の保存に重きをおいていた一方、兩班は術数的数学や西欧数学研究に貪欲であり、両者の間には溝があった。
    李氏朝鮮時代は日本や清による侵略、キリスト教の典礼問題があったため攘夷の動きがあった。一方で時憲暦に関してのみ積極的に西洋の学問を探求した。
    
    \section*{第7回}
    \hspace{1em}中世から近代にかけて、琉球王国は貿易の中間点であったため、情報・物・人の往来が頻繁に行われていた。
    琉球王国は明に入貢し中国の朝貢体制に組み込まれる一方、島津氏の琉球侵攻により日本による関節統治が行われるなど異質な統治体制を敷かれた。 \\
    \hspace{1em}産業の面に目を配ると、琉球王国は砂糖の製造に力を入れていたことがわかる。 \\
    \hspace{1em}中国との交流の面では、中国との交流のために閩人三十六姓が活躍していたり、中国への留学生として官生が冊封使の船に便乗して派遣された。 \\
    \hspace{1em}さらに、日常の算術用に役人はそろばんを、庶民は藁算を使っていた。
    琉球王国では冊封体制の影響で中国の暦を使用することを強制されていた。
    しかし、年末までに中国から暦が届かなかったため、簡易的な暦を作成し届くまでは簡易版を使用していた。
    測量に関しては、印部石という測量基準の石を適切に設置して地図を作成していた様子が見られる。
    1800年以降は欧米列強との接触も起こった。
    
    \section*{第8回}
    \hspace{1em}近代以前の日本では、飛鳥時代に百済から暦法が伝来し導入した。
    その後陰陽寮が設置され暦の作成を担当するようになった。
    鎌倉時代になるとひらがなが混じった暦が作られるようになる。
    さらに室町時代、戦国時代と時代が進むにつれて地方での暦の需要が増加したため陰陽氏が地方に定住するようになった。
    しかし、地方暦により日付の食い違いが生じてしまう。 \\
    \hspace{1em}そこで、江戸幕府は暦の内容を統一する政策を実施する。
    江戸時代の改暦事業では授時暦をベースとして現行を作成した
    しかし日食の予報に失敗したため日中の緯度差を考慮して作り直した。
    こうしてできた貞享暦 が1685年に施行されることとなった。
    一方、近世にはすでに西欧天文学が伝来しており、望遠鏡も伝わってきていたが、この時点では望遠鏡を使用して暦が作られることはなかった。

    \section*{第9回}
    \hspace{1em}日本では江戸時代初期にはすでにそろばんが伝来していた形跡が残っている。
    詳細は不明だが、外国との貿易をする中で承認からそろばんの知識が伝来したと考えられている。
    このような中で鎖国以降市場が全国に展開されたことやインフラの整備、石高制の導入により計算技能が必要になったためそろばんが普及した。 \\
    \hspace{1em}そんな中1627年、吉田光由が塵劫記を出版した。
    塵劫記にはそろばんの使い方から実用的の問題、更には遊戯問題が掲載されており、当時のベストセラーとなった。
    しかし海賊版が横行したため、吉田は対策として12問の遺題を掲載した。
    すると数年のうちに遺題を解き、更には自ら遺題を残すものも現れた。
    これをきっかけに遺題継承がブームとなった。
    遺題は徐々に難易度を増し、ついには代数方程式を用いる必要がある問題が出てきた中で、天元術が再発見されることとなった。

    \section*{第10回}
    \hspace{1em}関孝和は給料計算や税務計算、国絵図作成の事務を生業としていた江戸時代の和算家である。
    関は傍書法という記号法を開発した。
    これにより天元術の式を複数の未知数にまで拡張できるようになり高次の連立方程式の回答を可能にし、多くの難問の解答を可能にした。
    さらに傍書法は天文学や測量術でも利用された。\\
    \hspace{1em}江戸時代、庶民は時間と金銭に余裕を持つことができたため、趣味として和算を学ぶ人が増加した。
    庶民は家元から指導を受け、一定のレベルに達すると免許状をもらうことができた。
    また、地方で和算を学ぶ人のために遊歴算家が地方に赴いて授業を行うといったこともなされていた。
    実学としても和算は使われており、地図の作成・暦・インフラ・財務計算など幅広く用いられていた。
    和算は明治時代の小学校解説に伴い公教育ではなくなったが、一部地域には和算塾が残った。

    \section*{第11回}
    \hspace{1em}17世紀には江戸幕府が主体となって国絵図が作成された。
    これは江戸幕府が大名への忠誠を示させ、大名の領地や石高、さらには軍事情報を把握するために17世紀だけでも4回は行われた。
    作成するのにかかる費用はすべて大名が負担していた。
    はじめは自分の領地を大きく見せるために縮尺を無視した地図が作成されていたが、後に規格が統一されていった。
    定められた規格に合致しない場l合はやり直しを命じられ、実裏面で不利になることがあった。
    このように国絵図を作らせたことで、結果的に全国に測量や和算の知識が普及することとなった。 \\
    \hspace{1em}このときに用いられた測量術は17世紀ヨーロッパから伝わったものである。
    様々な経路を通じて測量術が伝来し、17世紀後半になると17世紀後半には体系化されて阿蘭陀流として普及した。
    また、目標物までの距離を図る平板測量や地形を写し取るトラバース法も導入された。
    以上の技術を用いて国絵図が作成された。

    \section*{第12回}
    \hspace{1em}日本は16世紀より外国との交流を行ってきた。
    1640年代になると江戸幕府はオランダとの交流を行った。
    この貿易で流入した学術全般を蘭学と呼ぶ。
    日蘭貿易では砂糖や毛織物、書籍が輸入された。
    ここで輸入された書籍を経由して日本に科学技術がもたらされた。 \\
    \hspace{1em}オランダとの貿易は長崎の出島で行われ、オランダは東インド会社が常駐し、拠点としてオランダ商館が建てられた。
    江戸時代にはオランダ商館の館長が江戸時代を通じて166回も参府しており、また定宿などで非公式に日本人との交流も行われていた。
    18世紀になると阿蘭陀通詞とよばれる人々がオランダ人との仲介人になるなど民間の間での交流も増えた。 \\
    \hspace{1em}この転換期に将軍となったのが徳川吉宗である。
    吉宗は暦法や地図などに力を入れる一方、オランダ語の学習も命じていた。
    加えて、西洋の暦法を学ぶために漢訳洋書の輸入を緩和した。
    しかし、暦の改定を行うことは叶わなかった。

    \section*{第13回}
    \hspace{1em}1798年、伊能忠敬の師である高橋至時を中心に改暦が行われ、寛政暦が作られた。
    その後1803年ごろにラランデ暦書を入手し、これを利用して天保暦が作られた。
    このラランデ暦書は伊能忠敬の日本地図作成にも影響を与えた。
    また、このときの天文観測技術や天保暦は1839年の金環食の観測にもつながった。\\
    \hspace{1em}長崎の和蘭通詞である志筑忠雄は隠居後鎖国論や暦象新書などを著述し西洋の文化を日本に伝えた。
    志筑の弟子である馬場佐十郎は天文方に起用され、蕃書和解御用を併設し、在野の人間を積極的に起用しながら幕府の海外情報センターの役割を担った。
    またフランスの百科事典の翻訳も行った。 \\
    \hspace{1em}このように蘭学は発展したが、民間が興味本位で吸収しているのに対し、幕府などは対外問題や殖産興業の一環で取り入れているといった違いも見られた。
    大槻玄沢は解体新書の重訂版を編集した。
    またお雇い外国人のシーベルトはオランダ語や西洋医学を教えた。

    \section*{第14回}
    \hspace{1em}伊能忠敬は日本地図を作成したことで有名な人物である。
    伊能は師の高橋が入手したラランデ暦書書かれていた観測技術を利用して1800年から測量を行い1816年に測量を最後に地図を完成させた。 \\
    \hspace{1em}伊能は測量を行う際に導線法・交会法・経緯度決定のための天文観測を行った。
    計測したら下書きをした上で清書である大図を書き、大図を縮図連結させて中図を、さらに中図を縮図連結させて小図を作成した。 \\
    \hspace{1em}伊能の地図づくりでは現地の測量課たちがサポートして入った。
    その過程で、地方の測量家にも伊能の最新技術が伝わっていった。
    越中の石黒信由は石川県の地図を作成した人であるが、その作成に使われた器具は伊能と出会ったことで改良されたものであった。

    \section*{第15回}
    \hspace{1em}開陽丸とは幕末に幕府がオランダに発注した軍艦で、1868年に江差沖で座礁し沈没してしまった。
    この開陽丸は昭和49年の予備調査で発掘され、昭和50年から昭和59年にかけて発掘調査が行われた。
    発掘品の中には和算や暦学の本もあった。
    これは山路一郎が開陽丸に乗り込んだ際に積み込んだものと推測されている。 \\
    \hspace{1em}和算書の中には関孝和・武部賢弘が編集した大成算経や村瀬義益の遺題も存在した。
    大成算経は関の業績をほぼ網羅し、建部兄弟の最新の知見が記された総合和算書となっている。
    また暦学書では暦算全書と呼ばれる漢訳洋書の輸入解禁で輸入された本や、天文の観測記録をはじめ、日食・月食の予報が計算されている雑多な書類も含まれていた。

    %%%%%%%%%%%%%%%%%%%%%%%%%%%%%%%%%%%%%%%%%%%%%%%%%%%%%%%%%%%%%%%%%%%%%%
    \appendix
    \setcounter{figure}{0}
    \setcounter{table}{0}
    \numberwithin{equation}{section}
    \renewcommand{\thetable}{\Alph{section}\arabic{table}}
    \renewcommand{\thefigure}{\Alph{section}\arabic{figure}}
    %\def\thesection{付録\Alph{section}}
    \makeatletter 
    \newcommand{\section@cntformat}{付録 \thesection:\ }
    \makeatother
    %%%%%%%%%%%%%%%%%%%%%%%%%%%%%%%%%%%%%%%%%%%%%%%%%%%%%%%%%%%%%%%%%%%%%%
    
    
\end{document}