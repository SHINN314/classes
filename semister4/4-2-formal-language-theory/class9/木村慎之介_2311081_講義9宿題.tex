\documentclass[uplatex,dvipdfmx,a4paper,10pt]{jsarticle}
\usepackage{graphicx}
\usepackage{amsmath}
\usepackage{latexsym}
\usepackage{multirow}
\usepackage{url}
\usepackage[separate-uncertainty]{siunitx}
\usepackage{physics}
\usepackage{enumerate}
\usepackage{bm}
\usepackage{pdfpages}
\usepackage{pxchfon}

% \renewcommand{\rmdefault}{pplj}
% \renewcommand{\sfdefault}{phv}

\setlength{\textwidth}{165mm} %165mm-marginparwidth
\setlength{\marginparwidth}{40mm}
\setlength{\textheight}{225mm}
\setlength{\topmargin}{-5mm}
\setlength{\oddsidemargin}{-3.5mm}
% \setlength{\parindent}{0pt}

\def\vector#1{\mbox{\boldmath $#1$}}
\newcommand{\AmSLaTeX}{%
 $\mathcal A$\lower.4ex\hbox{$\!\mathcal M\!$}$\mathcal S$-\LaTeX}
\newcommand{\PS}{{\scshape Post\-Script}}
\def\BibTeX{{\rmfamily B\kern-.05em{\scshape i\kern-.025em b}\kern-.08em
 T\kern-.1667em\lower.7ex\hbox{E}\kern-.125em X}}
\newcommand{\DeLta}{{\mit\Delta}}
\renewcommand{\d}{{\rm d}}
\def\wcaption#1{\caption[]{\parbox[t]{100mm}{#1}}}
\def\rm#1{\mathrm{#1}}
\def\tempC{^\circ \rm{C}}

\makeatletter
\def\section{\@startsection {section}{1}{\z@}{-3.5ex plus -1ex minus -.2ex}{2.3ex plus .2ex}{\Large\bf}}
\def\subsection{\@startsection {subsection}{2}{\z@}{-3.25ex plus -1ex minus -.2ex}{1.5ex plus .2ex}{\normalsize\bf}}
\def\subsubsection{\@startsection {subsubsection}{3}{\z@}{-3.25ex plus -1ex minus -.2ex}{1.5ex plus .2ex}{\small\bf}}
\makeatother

\makeatletter
\def\@seccntformat#1{\@ifundefined{#1@cntformat}%
   {\csname the#1\endcsname\quad}%      default
   {\csname #1@cntformat\endcsname}%    enable individual control
}
\makeatother

\newcommand{\tenexp}[2]{#1\times10^{#2}}


\begin{document}
    % タイトル
    \begin{center}
        {\Large{\bf 形式言語理論 講義9宿題}} \\
        {\bf 2311081 木村慎之介} \\
    \end{center}

    \section*{Pumping Lemma 2の証明}
        \hspace{1em}言語\(L\)が正則であると仮定する。
        この時\(L\)を受理するDFA \(A=(Q, \Sigma, \delta, q_0, F)\)が存在する。
        ここで\(c = |Q| \geq 1\)とし、任意に\(w_1, w_2, w_3 \in \Sigma^{*}\)を取ってくる。
        なお、\(|Q| \geq 1\)であることは初期状態\(q_0\)が必ず\(Q\)に含まれることから明らか。 \\
        \indent さて、文字列\(w_1w_2w_3 = a_{11} \cdots a_{1m}a_{21} \cdots a_{2n}a_{31} \cdots a_{3l}\ (\text{ただし}w_1 = a_{11} \cdots a_{1m},\ w_2 = a_{21} \cdots a_{2n},\ w_3 = a_{31} \cdots a_{3l},\ a_{11}, \cdots ,a_{3l} \in \Sigma)\)を受理する過程は、状態\(q_0, q_{11}, \cdots, q_{1m}, q_{21}, \cdots, q_{2n}, q_{31}, \cdots, q_{3l-1} \in Q\)と受理状態\(q_{3l} \in F\)を用いて\(q_0 \xrightarrow{a_{11}} q_{11} \xrightarrow{a_{12}} \cdots \xrightarrow{a_{1m}} q_{1m} \xrightarrow{a_{21}} q_{21} \xrightarrow{a_{22}} \cdots \xrightarrow{a_{2n}} q_{2n} \xrightarrow{a_{31}} q_{31} \xrightarrow{a_{32}} \cdots \xrightarrow{a_{3l}} q_{3l}\)と表せる。
        ここで\(|w_2| = n \geq c\)の時、鳩の巣原理より\(q_{2i} = q_{2j}\)となる\(1 \leq i < j \leq n\)が存在する。 \\
        \indent 以上の議論を踏まえて\(u, x, v\)をそれぞれ\(u = a_{21} \cdots a_{2i}\ , x = a_{2\ i+1} \cdots a_{2j}\ ,v = a_{2\ j+1} \cdots a_{3l}\)とする。
        ただし、\(u, x, v\)は\(\hat{\delta}(u, q_{1, m}) = q_{2i} \ , \hat{\delta}(x, q_i) = q_{2i} \ , \hat{\delta}(v, q_{2i}) = q_{2n}\)を満たすことに注意する。
        \(u, x, v\)を以上のように持ってくると、\(w_2 = a_{21} \cdots a_{2i} a_{2\ i+1} \cdots a_{2j} a_{2\ j+1} \cdots a_{2n} = uxv\)と表せる。
        このとき\(w_1w_2w_3 \in L\)を仮定すると、任意の\(t \geq 0\)に対して\(\hat{\delta}(x, q_{2i}) = q_{2i}\)という性質から\(\hat{\delta}(x^t, q_i) = q_i\)となるので、初期状態から\(w_1ux^tvw_3\)を入力として受け取ると次のように状態が遷移する。
        
        \begin{align*}
            \hat{\delta}(w_1ux^tvw_3\ , q_0) &= \hat{\delta}(ux^tvw_3\ , \hat{\delta}(w_1 ,q_0) = q_{1m}) \\
                                             &= \hat{\delta}(x^tvw_3\ , \hat{\delta}(u, q_{1m}) = q_{2i}) \\
                                             &= \hat{\delta}(vw_3\ , \hat{\delta}(x^t, q_{2i}) = q_{2i}) \\
                                             &= \hat{\delta}(w_3\ , \hat{\delta}(v, q_{2i}) = q_{2n}) \\
                                             &= q_{3l}
        \end{align*}

        よって\(q_{3l} \in F\)であるため\(\forall t \geq 0\ [\ w_1ux^tvw_3 \in L\ ]\)が示された。
        逆に\(\forall t \geq 0\ [\ w_1ux^tvw_3 \in L\ ]\)を仮定する。
        ここで、\(t = 1\)とすれば\(u, x, v\)の定義から\(w_1ux^1vw_3 = w_1w_2w_3\)となる。
        仮定より\(w_1ux^1vw_3 = w_1w_2w_3 \in L\)となる。 \\
        \indent また\(x\)は\(1 \leq i < j\)であることから\(|x| \geq 1\)となり\(x \neq \lambda\)が示される。 \\
        \hspace{1em}以上よりPumping Lemma 2が証明された。

    %参考文献
    % \begin{thebibliography}{99}
    %     \bibitem{bi:1} これこれ
    % \end{thebibliography}
    
    
    %%%%%%%%%%%%%%%%%%%%%%%%%%%%%%%%%%%%%%%%%%%%%%%%%%%%%%%%%%%%%%%%%%%%%%
    \appendix
    \setcounter{figure}{0}
    \setcounter{table}{0}
    \numberwithin{equation}{section}
    \renewcommand{\thetable}{\Alph{section}\arabic{table}}
    \renewcommand{\thefigure}{\Alph{section}\arabic{figure}}
    %\def\thesection{付録\Alph{section}}
    \makeatletter 
    \newcommand{\section@cntformat}{付録 \thesection:\ }
    \makeatother
    %%%%%%%%%%%%%%%%%%%%%%%%%%%%%%%%%%%%%%%%%%%%%%%%%%%%%%%%%%%%%%%%%%%%%%
    
    
\end{document}