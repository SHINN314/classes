\documentclass[uplatex,dvipdfmx,a4paper,10pt]{jsarticle}
\usepackage{graphicx}
\usepackage{amsmath}
\usepackage{latexsym}
\usepackage{multirow}
\usepackage{url}
\usepackage[separate-uncertainty]{siunitx}
\usepackage{physics}
\usepackage{enumerate}
\usepackage{bm}
\usepackage{pdfpages}
\usepackage{pxchfon}

% \renewcommand{\rmdefault}{pplj}
% \renewcommand{\sfdefault}{phv}

\setlength{\textwidth}{165mm} %165mm-marginparwidth
\setlength{\marginparwidth}{40mm}
\setlength{\textheight}{225mm}
\setlength{\topmargin}{-5mm}
\setlength{\oddsidemargin}{-3.5mm}
\setlength{\parindent}{1em}

\def\vector#1{\mbox{\boldmath $#1$}}
\newcommand{\AmSLaTeX}{%
 $\mathcal A$\lower.4ex\hbox{$\!\mathcal M\!$}$\mathcal S$-\LaTeX}
\newcommand{\PS}{{\scshape Post\-Script}}
\def\BibTeX{{\rmfamily B\kern-.05em{\scshape i\kern-.025em b}\kern-.08em
 T\kern-.1667em\lower.7ex\hbox{E}\kern-.125em X}}
\newcommand{\DeLta}{{\mit\Delta}}
\renewcommand{\d}{{\rm d}}
\def\wcaption#1{\caption[]{\parbox[t]{100mm}{#1}}}
\def\rm#1{\mathrm{#1}}
\def\tempC{^\circ \rm{C}}

\makeatletter
\def\section{\@startsection {section}{1}{\z@}{-3.5ex plus -1ex minus -.2ex}{2.3ex plus .2ex}{\Large\bf}}
\def\subsection{\@startsection {subsection}{2}{\z@}{-3.25ex plus -1ex minus -.2ex}{1.5ex plus .2ex}{\normalsize\bf}}
\def\subsubsection{\@startsection {subsubsection}{3}{\z@}{-3.25ex plus -1ex minus -.2ex}{1.5ex plus .2ex}{\small\bf}}
\makeatother

\makeatletter
\def\@seccntformat#1{\@ifundefined{#1@cntformat}%
   {\csname the#1\endcsname\quad}%      default
   {\csname #1@cntformat\endcsname}%    enable individual control
}
\makeatother

\newcommand{\tenexp}[2]{#1\times10^{#2}}


\begin{document}
    % タイトル
    \begin{center}
        {\Large{\bf 形式言語理論 講義4課題}} \\
        {\bf 2311081 木村慎之介} \\
    \end{center}

    \section*{Subset constructionを利用してNFAからDFAを作る}
      \hspace{1em}はじめにDFAの状態遷移表を作る過程を説明する。\\ \indent
      まず初期状態$\{q_0\}$から入力$0$、$1$をそれぞれを受け取ったときの遷移先の状態を
      調べる。遷移先の状態は現在の状態(ここでは初期状態$\{q_0\}$)の全要素に対して入力を
      受け取ったときに遷移可能な状態をすべて集めた集合とする。
      もし遷移先の状態の遷移先についてすでに調べていたら、そこで操作を終了する。そうでなければ
      初期状態の遷移先を調べたときと同じように遷移先の状態の遷移先について調べる。
      以降、再帰的に調べていない遷移先の状態がでてくるたびに上記の方法で遷移先の状態の遷移先を調べる。\\ \indent
      以上の方法でDFAの状態遷移表を作ると次のようになる。なお、受理状態に関してはDFAの状態の要素に一つでも
      NFAの受理状態が存在すれば、そのDFAの状態を受理状態とする。

      \begin{table}[h]
        \begin{center}
          \caption{状態遷移表}
          \begin{tabular}{rlll} \hline
            状態 & $0$ & $1$ \\ \hline
              $\rightarrow$$\{q_0\}$ & $\{q_0\}$ & $\{q_0, q_1\}$  \\
              $\{q_0, q_1\}$ & $\{q_0, q_2\}$ & $\{q_0, q_1, q_2\}$ \\
              $\{q_0, q_2\}$ & $\{q_0, q_3\}$ & $\{q_0, q_1, q_3\}$ \\
              $\ast$$\{q_0, q_3\}$ & $\{q_0\}$ & $\{q_0, q_1\}$  \\
              $\{q_0, q_1, q_2\}$ & $\{q_0, q_2, q_3\}$ & $\{q_0, q_1, q_2, q_3\}$  \\
              $\ast$$\{q_0, q_1, q_3\}$ & $\{q_0, q_2\}$ & $\{q_0, q_1, q_2\}$ \\
              $\ast$$\{q_0, q_2, q_3\}$ & $\{q_0, q_3\}$ & $\{q_0, q_1, q_3\}$ \\
              $\ast$$\{q_0, q_1, q_2, q_3\}$ & $\{q_0, q_2, q_3\}$ & $\{q_0, q_1, q_3\}$ \\ \hline
            \end{tabular}
        \end{center}
      \end{table}

      したがって、以上の表から作られるDFAがNFAと同じ言語を受理するDFAとなる。
    
    %参考文献
    % \begin{thebibliography}{99}
    %     \bibitem{bi:1} これこれ
    % \end{thebibliography}
    
    
    %%%%%%%%%%%%%%%%%%%%%%%%%%%%%%%%%%%%%%%%%%%%%%%%%%%%%%%%%%%%%%%%%%%%%%
    \appendix
    \setcounter{figure}{0}
    \setcounter{table}{0}
    \numberwithin{equation}{section}
    \renewcommand{\thetable}{\Alph{section}\arabic{table}}
    \renewcommand{\thefigure}{\Alph{section}\arabic{figure}}
    %\def\thesection{付録\Alph{section}}
    \makeatletter 
    \newcommand{\section@cntformat}{付録 \thesection:\ }
    \makeatother
    %%%%%%%%%%%%%%%%%%%%%%%%%%%%%%%%%%%%%%%%%%%%%%%%%%%%%%%%%%%%%%%%%%%%%%
    
    
\end{document}