\section{選択問題}
\subsection{問1}
\hspace{1em}許容解が存在しない最適化問題(最小化問題)として以下の例が考えられる。

\begin{equation}
    \left\{
        \begin{aligned}
            &   \text{目的関数: } f(x, y) = x + y \\
            &   \text{条件: } x \geq 0 \land x < 0
        \end{aligned}
    \right.\
\end{equation}

\noindent この最適化問題では条件をみたす\(x \in \mathbb{R}\)が存在しないため、題意をみたす最適化問題となっている。

\subsection{問2}
\hspace{1em}許容解は存在するが、最適解が存在しない最適化問題は存在する。
例として、以下の最適化問題(最小化問題)が挙げられる。

\begin{equation}
    \left\{
        \begin{aligned}
            &   \text{目的関数: } f(x) = \frac{1}{x} \\
            &   \text{条件: } S = \{ x \in \mathbb{R} | x \geq 1\}
        \end{aligned}
    \right.\
\end{equation}

この最適化問題の許容解としては\(x = 1\)などが挙げられる。
しかし、どのような\(x \in S\)に対しても\(f(x) > f(x + 1)\)となる。
つまり\(f(x)\)よりも小さい目的関数値が存在するため最適解は存在しない。

\subsection{問3}
\hspace{1em}最適解は存在するが、許容解が存在しない最適化問題は存在しない。
なぜならば、最適解の定義から最適解自信が許容解となっているからである。