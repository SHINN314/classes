\documentclass[uplatex,dvipdfmx,a4paper,10pt]{jsarticle}
\usepackage{graphicx}
\usepackage{amsmath}
\usepackage{latexsym}
\usepackage{multirow}
\usepackage{url}
\usepackage[separate-uncertainty]{siunitx}
\usepackage{physics}
\usepackage{enumerate}
\usepackage{bm}
\usepackage{pdfpages}
\usepackage{pxchfon}
\usepackage{here}
\usepackage{circledsteps}

% \renewcommand{\rmdefault}{pplj}
% \renewcommand{\sfdefault}{phv}

\setlength{\textwidth}{165mm} %165mm-marginparwidth
\setlength{\marginparwidth}{40mm}
\setlength{\textheight}{225mm}
\setlength{\topmargin}{-5mm}
\setlength{\oddsidemargin}{-3.5mm}
% \setlength{\parindent}{0pt}

\def\vector#1{\mbox{\boldmath $#1$}}
\newcommand{\AmSLaTeX}{%
 $\mathcal A$\lower.4ex\hbox{$\!\mathcal M\!$}$\mathcal S$-\LaTeX}
\newcommand{\PS}{{\scshape Post\-Script}}
\def\BibTeX{{\rmfamily B\kern-.05em{\scshape i\kern-.025em b}\kern-.08em
 T\kern-.1667em\lower.7ex\hbox{E}\kern-.125em X}}
\newcommand{\DeLta}{{\mit\Delta}}
\renewcommand{\d}{{\rm d}}
\def\wcaption#1{\caption[]{\parbox[t]{100mm}{#1}}}
\def\rm#1{\mathrm{#1}}
\def\tempC{^\circ \rm{C}}

\makeatletter
\def\section{\@startsection {section}{1}{\z@}{-3.5ex plus -1ex minus -.2ex}{2.3ex plus .2ex}{\Large\bf}}
\def\subsection{\@startsection {subsection}{2}{\z@}{-3.25ex plus -1ex minus -.2ex}{1.5ex plus .2ex}{\normalsize\bf}}
\def\subsubsection{\@startsection {subsubsection}{3}{\z@}{-3.25ex plus -1ex minus -.2ex}{1.5ex plus .2ex}{\small\bf}}
\makeatother

\makeatletter
\def\@seccntformat#1{\@ifundefined{#1@cntformat}%
   {\csname the#1\endcsname\quad}%      default
   {\csname #1@cntformat\endcsname}%    enable individual control
}
\makeatother

\newcommand{\tenexp}[2]{#1\times10^{#2}}


\begin{document}
    % タイトル
    \begin{center}
        {\Large{\bf オペレーションズリサーチ基礎 第1回課題}} \\
        {\bf 2311081 木村慎之介} \\
    \end{center}

    目的関数は\(x_0=x_1+x_2+2x_3\)である。これを最小化したい。制約条件は
\begin{eqnarray}
  x_2 + 2x_3 = 5 \\
  -x_1-3x_2+x_3\geq -5 \\
  x_1+x_2-3x_3 \leq -2 \\
  x_1 + 2x_2 + 2x_3 \geq 2 \\
  x_1\geq 0, x_2\geq 0, x_3\geq 0
\end{eqnarray}
である。
\newline 式(1)に非負の人為変数\(x_4\)を加える。
\newline 式(2)の両辺に\(-1\)をかけて、非負のスラック変数\(x_5\) を加える。
\newline 式(3)の両辺に\(-1\)をかけて、非負のサープラス変数\(x_6\) を引き、非負の人為変数\(x_7\)を加える。
\newline 式(4)の両辺から非負のサープラス変数\(x_8\) を引き、非負の人為変数\(x_9\)を加える。
\newline 罰金法で解く。目的関数に人為変数の\(M\)倍を加える。目的関数、制約条件は以下のようになる。
\begin{eqnarray}
  目的関数 \hspace{2pt} &x_0 =x_1+x_2+2x_3\ + Mx_4 + Mx_7 + Mx_9  \rightarrow 最小化\\
  &x_2 + 2x_3 + x_4 = 5 \\
  &x_1+3x_2-x_3+x_5 = 5 \\
  &-x_1-x_2+3x_3 -x_6 + x_7 =2 \\
  &x_1 + 2x_2 + 2x_3 -x_8 + x_9 = 2 \\
  &x_i\geq 0 (i = 1,2,\cdots, 8, 9)
\end{eqnarray}
\begin{table}[H]
  \centering
  \fontsize{30pt}{30pt}
  \large
  \begin{tabular}{|c|c|c|c|c|c|c|c|c|c|c|c|c|}
\hline
 & 係数 & 1& 1 & 2 & \(M\) & 0 & 0 & \(M\) & 0 & \(M\) & \multicolumn{2}{c|}{} \\
\hline
係数 & 基底変数  & \(x_1\) & \(x_2\) & \(x_3\) & \(x_4\) & \(x_5\) & \(x_6\) & \(x_7\) & \(x_8\) & \(x_9\) & 定数項 & \(\theta \) \\
\hline
\(M\) & \(x_4\) & 0 & 1 & 2 & 1 & 0 & 0 & 0 & 0 & 0 &  5 & \(\frac{5}{2}\)\\
\hline
0 & \(x_5\) & 1 & 3 & -1 & 0 & 1 & 0 & 0 & 0 & 0 &  5 & \(\infty\)\\
\hline
\(M\) & \(x_7\) & -1 & -1 & \circled{3} & 0 & 0 & -1 & 1 & 0 & 0 &  2 & \(\frac{2}{3}\)\\
\hline
\(M\) & \(x_9\) & 1 & 2 & 2 & 0 & 0 & 0 & 0 & -1 & 1 &  2 & \(1\)\\
\hline
\multicolumn{2}{|c|}{ \(\pi_j\)} & 0 & 2\(M\) & 7\(M\) & \(M\) & 0 & \(-M\) & \(M\) & \(-M\) & \(M\) & \(9M\) & \\
\hline
\multicolumn{2}{|c|}{ \(c_j-\pi_j\)} & 1 & \(1-2M\) & \circled{\(2-7M\)} & 0 & 0 & \(M\) & 0 & \(M\) & 0 & & \\
 \hline
\end{tabular}
\end{table}

\begin{table}[H]
  \centering
  \fontsize{30pt}{30pt}
  \small
  \begin{tabular}{|c|c|c|c|c|c|c|c|c|c|c|c|c|}
\hline
 & 係数 & 1& 1 & 2 & \(M\) & 0 & 0 & \(M\) & 0 & \(M\) & \multicolumn{2}{c|}{} \\
\hline
係数 & 基底変数  & \(x_1\) & \(x_2\) & \(x_3\) & \(x_4\) & \(x_5\) & \(x_6\) & \(x_7\) & \(x_8\) & \(x_9\) & 定数項 & \(\theta \) \\
\hline
\(M\) & \(x_4\) & \(\frac{2}{3}\) & \(\frac{5}{3}\) & 0 & 1 & 0 & \(\frac{2}{3}\) & \(-\frac{2}{3}\) & 0 & 0 &  \(\frac{11}{3}\) &  \(\frac{11}{5}\)\\
\hline
0 & \(x_5\) & \(\frac{2}{3}\) & \(\frac{8}{3}\) & 0 & 0 & 1 & \(-\frac{1}{3}\) & \(\frac{1}{3}\) & 0 & 0 &  \(\frac{17}{3}\) & \(\frac{17}{8}\)\\

\hline
\(2\) & \(x_3\) & \(-\frac{1}{3}\) & \(-\frac{1}{3}\) & 1 & 0 & 0 & \(-\frac{1}{3}\) & \(\frac{1}{3}\) & 0 & 0 & \(\frac{2}{3}\) & \(\infty\)\\

\hline
\(M\) & \(x_9\) & \(\frac{5}{3}\) & \circled{\(\frac{8}{3}\)} & 0 & 0 & 0 & \(\frac{2}{3}\) & \(-\frac{2}{3}\) & -1 & 1 &  \(\frac{2}{3}\) & \(\frac{1}{4}\) \\
 \hline
\multicolumn{2}{|c|}{ \(\pi_j\)} & \(\frac{7}{3}M-\frac{2}{3}\) & \(\frac{13}{3}M-\frac{2}{3}\) & 2 & \(M\) & 0 & \(\frac{4}{3}M-\frac{2}{3}\) & \(-\frac{4}{3}M+\frac{2}{3}\) & \(-M\) & \(M\) &  \(\frac{13}{3}M+\frac{4}{3}\) &\\
\hline
\multicolumn{2}{|c|}{ \(c_j-\pi_j\)} & \(-\frac{7}{3}M+\frac{5}{3}\) & \circled{ \(-\frac{13}{3}M+\frac{5}{3}\)}  & 0 & 0 & 0 & \(-\frac{4}{3}M+\frac{2}{3}\) & \(\frac{7}{3}M-\frac{2}{3}\) & \(M\) & 0 & & \\
 \hline
\end{tabular}
\end{table}


\begin{table}[H]
  \centering
  \fontsize{30pt}{30pt}
  \small
  \begin{tabular}{|c|c|c|c|c|c|c|c|c|c|c|c|c|}
\hline
 & 係数 & 1& 1 & 2 & \(M\) & 0 & 0 & \(M\) & 0 & \(M\) & \multicolumn{2}{c|}{} \\
\hline
係数 & 基底変数  & \(x_1\) & \(x_2\) & \(x_3\) & \(x_4\) & \(x_5\) & \(x_6\) & \(x_7\) & \(x_8\) & \(x_9\) & 定数項 & \(\theta \) \\
\hline
\(M\) & \(x_4\) & \(-\frac{3}{8}\) & 0 & 0 & 1 & 0 & \(\frac{1}{4}\) & \(-\frac{1}{4}\) & \(\frac{5}{8}\) & \(-\frac{5}{8}\) &  \(\frac{13}{4}\) &  \(\frac{26}{5}\) \\
\hline
0 & \(x_5\) & \(-1\) &0 & 0 & 0 & 1 & \(-1\) & 1 & \circled{1} & \(-1\) &  5 &  5 \\

\hline
\(2\) & \(x_3\) & \(-\frac{3}{8}\) & 0 & 3 & 0 & 0 & \(-\frac{3}{4}\) & \(\frac{3}{4}\) & \(-\frac{3}{8}\) & \(\frac{3}{8}\) & \(\frac{9}{4}\) & \(\infty\)\\

\hline
\(1\) & \(x_2\) & \(\frac{5}{8}\) & 1 & 0 & 0 & 0 & \(\frac{1}{4}\) & \(-\frac{1}{4}\) & \(-\frac{3}{8}\) & \(\frac{3}{8}\) &  \(\frac{1}{4}\) & \(\infty\) \\
 \hline

\multicolumn{2}{|c|}{ \(\pi_j\)} & \(-\frac{3}{8}M-\frac{1}{8}\) & 1 & 6 & \(M\) & 0 & \(\frac{1}{4}M-\frac{5}{4}\) & \(-\frac{1}{4}M+\frac{5}{4}\) &  \(\frac{5}{8}M-\frac{9}{8}\) &  \(-\frac{5}{8}M+\frac{9}{8}\)  &  \(\frac{13}{4}M+\frac{19}{4}\) &\\
\hline
\multicolumn{2}{|c|}{ \(c_j-\pi_j\)} & \(\frac{3}{8}M+\frac{9}{8}\) & 0 & \(-4\) & 0 & 0 & \(-\frac{1}{4}M+\frac{5}{4}\) & \(\frac{5}{4}M-\frac{5}{4}\) & \circled{ \(-\frac{5}{8}M+\frac{9}{8}\)} &  \(\frac{13}{8}M-\frac{9}{8}\)  &  \(-\frac{13}{4}M-\frac{19}{4}\) &\\

 \hline
\end{tabular}
\end{table}


\begin{table}[H]
  \centering
  \fontsize{30pt}{30pt}
  \small
  \begin{tabular}{|c|c|c|c|c|c|c|c|c|c|c|c|c|}
\hline
 & 係数 & 1& 1 & 2 & \(M\) & 0 & 0 & \(M\) & 0 & \(M\) & \multicolumn{2}{c|}{} \\
\hline
係数 & 基底変数  & \(x_1\) & \(x_2\) & \(x_3\) & \(x_4\) & \(x_5\) & \(x_6\) & \(x_7\) & \(x_8\) & \(x_9\) & 定数項 & \(\theta \) \\
\hline
\(M\) & \(x_4\) & \(\frac{2}{5}\) & 0 & 0 & \(\frac{8}{5}\) & \(-1\) & \circled{\(\frac{7}{5}\)} & \(-\frac{7}{5}\) & 0& 0 &  \(\frac{1}{5}\) &  \(\frac{1}{7}\) \\
\hline
0 & \(x_8\) & \(-1\) &0 & 0 & 0 & 1 & \(-1\) & 1 & 1 & \(-1\) &  5 &  \(\infty\) \\

\hline
\(2\) & \(x_3\) & \(-2\) & 0 & 8 & 0 & 1 & \(-3\) & 3 & 0 & 0 & 11 & \(\infty\)\\

\hline
\(1\) & \(x_2\) & \(\frac{2}{3}\) & \(\frac{8}{3}\) & 0 & 0 & 1 & \(-\frac{1}{3}\) & \(\frac{1}{3}\) & 0 & 0 &  \(\frac{17}{3}\) & \(\infty\) \\
 \hline

\multicolumn{2}{|c|}{ \(\pi_j\)} & \(-\frac{2}{5}M-\frac{10}{3}\) & \(\frac{8}{3}\) & 16 & \(\frac{8}{5}M\)& \(3-M\) & \(\frac{7}{5}M-\frac{19}{3}\) & \(-\frac{7}{5}M+\frac{19}{3}\) & 0 & 0 &  \(\frac{1}{5}M+\frac{83}{3}\) &\\
\hline
\multicolumn{2}{|c|}{ \(c_j-\pi_j\)} & \(-\frac{2}{5}M+\frac{13}{3}\) & \(-\frac{5}{3}\) & \(-14\) & \(-\frac{3}{5}M\) & \(M-3\) & \circled{\(-\frac{7}{5}M+\frac{19}{3}\)} & \(\frac{12}{5}M-\frac{19}{3}\) & 0 & \(M\)  &  &\\

 \hline
\end{tabular}
\end{table}


\begin{table}[H]
  \centering
  \fontsize{30pt}{30pt}
  \small
  \begin{tabular}{|c|c|c|c|c|c|c|c|c|c|c|c|c|}
\hline
 & 係数 & 1& 1 & 2 & \(M\) & 0 & 0 & \(M\) & 0 & \(M\) & \multicolumn{2}{c|}{} \\
\hline
係数 & 基底変数  & \(x_1\) & \(x_2\) & \(x_3\) & \(x_4\) & \(x_5\) & \(x_6\) & \(x_7\) & \(x_8\) & \(x_9\) & 定数項 & \(\theta \) \\
\hline
0 & \(x_6\) & \(\frac{2}{7}\) & 0 & 0 & \(\frac{8}{7}\) &  \(-\frac{5}{7}\)  & 1 & \(-1\) & 0& 0 &  \(\frac{1}{7}\) &  \(\infty\) \\
\hline
0 & \(x_8\) &  \(-\frac{5}{7}\)  &0 & 0 &  \(\frac{8}{7}\)  &  \(\frac{2}{7}\)  & 0 & 0 & 1 & \(-1\) &   \(\frac{36}{7}\)  &  \(\infty\) \\

\hline
\(2\) & \(x_3\) &  \(-\frac{8}{7}\)  & 0 & \circled{8} &  \(\frac{24}{7}\)  &  \(-\frac{8}{7}\)  & 0 & 0 & 0 & 0 &  \(\frac{80}{7}\)  & \(\frac{10}{7}\)\\

\hline
\(1\) & \(x_2\) & \(\frac{16}{7}\) & 8 & 0 &  \(\frac{8}{7}\)  &  \(\frac{16}{7}\)  & 0 & 0 & 0 & 0 &  \(\frac{120}{7}\) & \(\infty\) \\
 \hline

\multicolumn{2}{|c|}{ \(\pi_j\)} & 0 & 0 & 16 & 8 & 0 & 0 & 0 & 0 & 0 &  40 &\\
\hline
\multicolumn{2}{|c|}{ \(c_j-\pi_j\)} & 1 & \(-7\)& \circled{\(-14\)} & \(M-8\) & 0 & 0 & \(M\) & 0 & \(M\)  &  &\\

 \hline
\end{tabular}
\end{table}


\begin{table}[H]
  \centering
  \fontsize{30pt}{30pt}
  \small
  \begin{tabular}{|c|c|c|c|c|c|c|c|c|c|c|c|c|}
\hline
 & 係数 & 1& 1 & 2 & \(M\) & 0 & 0 & \(M\) & 0 & \(M\) & \multicolumn{2}{c|}{} \\
\hline
係数 & 基底変数  & \(x_1\) & \(x_2\) & \(x_3\) & \(x_4\) & \(x_5\) & \(x_6\) & \(x_7\) & \(x_8\) & \(x_9\) & 定数項 & \(\theta \) \\
\hline
0 & \(x_6\) & \(\frac{2}{7}\) & 0 & 0 & \(\frac{8}{7}\) &  \(-\frac{5}{7}\)  & 1 & \(-1\) & 0& 0 &  \(\frac{1}{7}\) &  \(\infty\) \\
\hline
0 & \(x_8\) &  \(-\frac{5}{7}\)  &0 & 0 &  \(\frac{8}{7}\)  &  \(\frac{2}{7}\)  & 0 & 0 & 1 & \(-1\) &   \(\frac{36}{7}\)  &  \(\infty\) \\

\hline
\(2\) & \(x_3\) &  \(-\frac{1}{7}\)  & 0 & 1 &  \(\frac{3}{7}\)  &  \(-\frac{1}{7}\)  & 0 & 0 & 0 & 0 &  \(\frac{10}{7}\)  & \(\infty\)\\

\hline
\(1\) & \(x_2\) & \(\frac{16}{7}\) & \circled{8} & 0 &  \(\frac{8}{7}\)  &  \(\frac{16}{7}\)  & 0 & 0 & 0 & 0 &  \(\frac{120}{7}\) & \(\frac{15}{7}\) \\
 \hline

\multicolumn{2}{|c|}{ \(\pi_j\)} & 2 & 8 & 2 & 2 & 2 & 0 & 0 & 0 & 0 & 20  &\\
\hline
\multicolumn{2}{|c|}{ \(c_j-\pi_j\)} & \(-1\) & \circled{\(-7\)}& 0 & \(M-2\) & \(-2\) & 0 & \(M\) & 0 & \(M\)  &  &\\

 \hline
\end{tabular}
\end{table}


\begin{table}[H]
  \centering
  \fontsize{30pt}{30pt}
  \small
  \begin{tabular}{|c|c|c|c|c|c|c|c|c|c|c|c|c|}
\hline
 & 係数 & 1& 1 & 2 & \(M\) & 0 & 0 & \(M\) & 0 & \(M\) & \multicolumn{2}{c|}{} \\
\hline
係数 & 基底変数  & \(x_1\) & \(x_2\) & \(x_3\) & \(x_4\) & \(x_5\) & \(x_6\) & \(x_7\) & \(x_8\) & \(x_9\) & 定数項 & \(\theta \) \\
\hline
0 & \(x_6\) & \(\frac{2}{7}\) & 0 & 0 & \(\frac{8}{7}\) &  \(-\frac{5}{7}\)  & 1 & \(-1\) & 0& 0 &  \(\frac{1}{7}\) &  \(\) \\
\hline
0 & \(x_8\) &  \(-\frac{5}{7}\)  &0 & 0 &  \(\frac{8}{7}\)  &  \(\frac{2}{7}\)  & 0 & 0 & 1 & \(-1\) &   \(\frac{36}{7}\)  &  \(\) \\

\hline
\(2\) & \(x_3\) &  \(-\frac{1}{7}\)  & 0 & 1 &  \(\frac{3}{7}\)  &  \(-\frac{1}{7}\)  & 0 & 0 & 0 & 0 &  \(\frac{10}{7}\)  & \(\)\\

\hline
\(1\) & \(x_2\) & \(\frac{2}{7}\) & 1 & 0 &  \(\frac{1}{7}\)  &  \(\frac{2}{7}\)  & 0 & 0 & 0 & 0 &  \(\frac{15}{7}\) & \(\) \\
 \hline

\multicolumn{2}{|c|}{ \(\pi_j\)} & 0 & 1 & 2 & 1 & 0 & 0 & 0 & 0 & 0 & 5  &\\
\hline
\multicolumn{2}{|c|}{ \(c_j-\pi_j\)} & 1 & 0 & 0 & \(M-1\) & 0 & 0 & \(M\) & 0 & \(M\)  &  &\\

 \hline
\end{tabular}
\end{table}

\(c_j-\pi_j\)が全部非負の値になったので、これ以上最適化できない。よって、目的関数は基底変数\(x_2=\frac{15}{7}, x_3 = \frac{10}{7}, x_6 = \frac{1}{7}, x_8 = \frac{36}{7} \)、非基底変数\(x_1,x_4,x_5,x_7,x_9=0\)のとき最小値5をとる。


 % \multicolumn{3}{c}{Class Distribution}


    %参考文献
    % \begin{thebibliography}{99}
    %     \bibitem{bi:1} これこれ
    % \end{thebibliography}
    
    
    %%%%%%%%%%%%%%%%%%%%%%%%%%%%%%%%%%%%%%%%%%%%%%%%%%%%%%%%%%%%%%%%%%%%%%
    \appendix
    \setcounter{figure}{0}
    \setcounter{table}{0}
    \numberwithin{equation}{section}
    \renewcommand{\thetable}{\Alph{section}\arabic{table}}
    \renewcommand{\thefigure}{\Alph{section}\arabic{figure}}
    %\def\thesection{付録\Alph{section}}
    \makeatletter 
    \newcommand{\section@cntformat}{付録 \thesection:\ }
    \makeatother
    %%%%%%%%%%%%%%%%%%%%%%%%%%%%%%%%%%%%%%%%%%%%%%%%%%%%%%%%%%%%%%%%%%%%%%
    
    
\end{document}