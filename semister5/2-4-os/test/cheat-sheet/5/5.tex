\section{記憶管理}
\subsection*{関数ポインタとセグフォ}
関数ポインタは読み込みのみが許されている記憶領域に保存されている。
そのためプログラム内でポインタを読み込むことはできるが、書き込もうとするとosによってセグメンテーション違反としてエラーが吐き出される。
\subsection*{仮想アドレス \(\backslash\) 論理アドレス}
アプリケーションがメモリにアクセスする時に使うアドレス。
実アドレスと一致するとは限らない。

\subsection*{実アドレス \(\backslash\) 物理アドレス}
実際の物理メモリ上のアドレス。

\subsection*{MMU}
仮想アドレスと実アドレスを対応付けるOSの機能。
ブロック単位で対応をつける。
\begin{itemize}
    \item ブロックサイズが固定: ページング
    \item ブロックサイズが可変: セグメンテーション
\end{itemize}