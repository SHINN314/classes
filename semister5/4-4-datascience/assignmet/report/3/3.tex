\section{問3}
\hspace{1em}ベイズの定理を用いて\(P(A_1)\)と\(P(A_2)\)の事後確率の分子を以下のように求めた。

\begin{align}
    P(A_1 | (1, 6, 1))  &\propto \frac{1}{2} \times \frac{3!}{2!0!0!0!0!1!} \times (\frac{1}{6})^2 (\frac{1}{6}) \\
                        &= \frac{1}{144} \\
    P(A_2 | (1, 6, 1))  &\propto \frac{1}{2} \times \frac{3!}{2!0!0!0!0!1!} \times (\frac{1}{2})^2 (\frac{1}{10}) \\
                        &= \frac{3}{80}
\end{align}

\hspace{1em}以上2つを正規化すると\(P(A_1 | (1, 6, 1)) = \frac{5}{32}\)、\(P(A_2 | (1, 6, 1)) = \frac{27}{32}\)と求まった。