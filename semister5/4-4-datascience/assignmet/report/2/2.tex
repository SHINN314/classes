\section{問2}
\subsection{(1)}
\hspace{1em}作成したプログラムを以下に乗せる。

\begin{lstlisting}[caption={最小二乗推定量と正則化最小二乗推定量を求めるプログラム}, label={code_regression}]
# Print the first 6 rows of the data used for analysis
print("--- 分析に使用するデータ(dat)の先頭6行 ---")
print(head(dat))

# OLS and Ridge regression
beta_ols <- solve(t(X) %*% X) %*% t(X) %*% y

rownames(beta_ols) <- c("Intercept", "x1", "x2", "x3", "x4", "x5", "x6")
colnames(beta_ols) <- "OLS_estimate"

lambda <- 1
p <- ncol(X)
I <- diag(p)

I[1, 1] <- 0

# Ridge regression
beta_ridge <- solve(t(X) %*% X + lambda * I) %*% t(X) %*% y

rownames(beta_ridge) <- c("Intercept", "x1", "x2", "x3", "x4", "x5", "x6")
colnames(beta_ridge) <- "Ridge_estimate"

# Output results
print("--- 最小二乗推定量 (OLS) ---")
print(beta_ols)

print("--- 正則化最小二乗推定量 (Ridge, λ=1) ---")
print(beta_ridge)
\end{lstlisting}

次に、実行結果を以下に乗せる。

\begin{lstlisting}[caption={コード\ref{code_regression}の実行結果}, label={code_output_regression}]
[1] "--- 分析に使用するデータ(dat)の先頭6行 ---"
                    x1         x2         x3         x4         x5        x6
Courtelary   0.8051305 -1.4820682  0.1062125 -0.7477267 0.77503669 0.8029583
Delemont     1.0372847 -0.2447942 -0.2057867  1.0477479 0.77503669 1.0359411
Franches-Mnt 1.7897846 -0.4825622 -0.6217858  1.2529998 0.08838778 1.7899671
Moutier      1.2534283 -0.6234617 -0.4137863 -0.1768099 0.12272023 1.2563525
Neuveville   0.5409551 -0.3152440  0.4182118 -0.8628212 0.22571757 0.5383002
Porrentruy   0.4769125 -0.6762990 -0.4137863  1.1851420 2.28566429 0.4798413
                       y
Courtelary   -0.18668632
Delemont     -1.31480509
Franches-Mnt -1.44015162
Moutier      -0.56272591
Neuveville    0.06400674
Porrentruy   -0.93876550
[1] "--- 最小二乗推定量 (OLS) ---"
           OLS_estimate
Intercept  3.760880e-15
x1        -2.708939e+01
x2        -2.317531e-01
x3         3.924185e-01
x4        -3.545175e-01
x5         3.419800e-02
x6         2.693602e+01
[1] "--- 正則化最小二乗推定量 (Ridge, λ=1) ---"
          Ridge_estimate
Intercept   1.266348e-16
x1         -8.695476e-02
x2         -2.450151e-01
x3          3.738593e-01
x4         -3.382417e-01
x5          3.664733e-02
x6         -8.079264e-02
\end{lstlisting}

結果を確認すると、\(x_1\)と\(x_6\)の係数において、最小二乗推定量では絶対値が大きくなっているのに対して、正則化最小二乗推定量は絶対値が他の回帰係数と差が無いことが読み取れる。
これは多重共線性が影響していると考えられる。

結果\ref{code_output_regression}の入力データの部分を見ると、\(x_1\)と\(x_6\)に相関関係が見られる。
相関が大きいデータがあると逆行列を求める際に、相関が高い部分の係数の値が大きくなってしまう。

一方、正則化最小二乗推定量ではこの相関が高くなってしまった時でも逆行列を求める際に、相関が高いところでもある程度値を抑える役割があるため、極端に大きな値にはならなかった。

\subsection{(2)}
以下に各回帰係数の検定を行うために実装したコードを乗せる。

\begin{lstlisting}[caption={回帰係数の検定プログラム}, label={code_check}]
# Remove the 6th column (x6) from the data
dat_rm <- dat[, -6]

# Prepare the explanatory variables as a matrix (x1 to x5)
x <- as.matrix(dat_rm[, 1:5])

# Run the linear model (lm) and output the summary of the results
model_summary <- summary(lm(y ~ x))

# Print the summary of the linear model
print(model_summary)
\end{lstlisting}

この出力結果は以下のようになった。

\begin{lstlisting}[caption={コード\ref{code_check}の実行結果}, label={code_output_check}]
Call:
lm(formula = y ~ x)

Residuals:
     Min       1Q   Median       3Q      Max 
-1.13350 -0.35713 -0.06654  0.38179  0.86331 

Coefficients:
              Estimate Std. Error t value Pr(>|t|)   
(Intercept)  1.352e-16  7.959e-02   0.000  1.00000   
xx1         -1.491e-01  1.467e-01  -1.016  0.31546   
xx2         -2.350e-01  1.249e-01  -1.881  0.06705 . 
xx3          3.982e-01  1.550e-01   2.569  0.01392 * 
xx4         -3.543e-01  1.102e-01  -3.215  0.00254 **
xx5          3.552e-02  9.236e-02   0.385  0.70250   
---
Signif. codes:  0 ‘***’ 0.001 ‘**’ 0.01 ‘*’ 0.05 ‘.’ 0.1 ‘ ’ 1

Residual standard error: 0.5457 on 41 degrees of freedom
Multiple R-squared:  0.7346,    Adjusted R-squared:  0.7022 
F-statistic:  22.7 on 5 and 41 DF,  p-value: 7.624e-11
\end{lstlisting}

まず実行結果のcoefficientsを確認するとPrがp値を求めていることが分かる。
ここを確認すると0.05より小さいのは\(x_3\)と\(x_4\)であることが分かる。

また、回帰全体の有効性を確認するには最後の行を確認すれば良い。
これによると、p値が0.05を大きく下回っているため少なくとも1つの回帰係数は有意差を持っていることが分かる。

\subsection{(3)}
以下に実装したプログラムを乗せる。